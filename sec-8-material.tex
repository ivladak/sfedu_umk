\setenumerate[1]{label={\arabic{enumi}.}}
\setenumerate[2]{label={\alph{enumii}.}}

\begin{enumerate}
\item Лекционные занятия:
\begin{enumerate}
\item аудитория, оснащенная маркерной доской, \item комплект электронных презентаций/слайдов, \item презентационная техника (проектор, экран, компьютер/ноутбук).
\end{enumerate}
\item Лабораторные работы: компьютерный класс
\begin{enumerate}
\item рабочее место преподавателя, оснащенное компьютером с доступом в Интернет, 
\item рабочие места студентов, предназначенные для работы в электронной образовательной среде. Рабочие станции студентов должны быть оснащены операционной системой GNU/Linux, текстовым редактором и компилятором/интерпретатором языка Haskell (например, GHC).
\end{enumerate}
\end{enumerate}


% \ssect[Учебно-лабораторное оборудование]
% Лекции проводятся в мультимедийном классе с презентационным оборудованием (проектором и экраном либо интерактивной доской). Лабораторные занятия проводятся в дисплейных классах с персональными компьютерами по числу, не уступающему числу студентов.

% 	\ssect[Программные средства]

% Компьютеры в дисплейных классах должны быть снабжены операционной системой GNU/Linux, желательно в виде дружественного к пользователю дистрибутива (например, Ubuntu Linux).

% На компьютерах в дисплейном классе должны быть распакованы в каталог \texttt{/bin} или \texttt{\textasciitilde/bin} три утилиты ассемблирования (as88/t88/s88) из сопроводительных материалов к учебнику Таненбаума (см. [1] в п.~\ref{online-res}). Для выполнения лабораторной работы по микропрограммированию необходимо наличие каталога Mic1MMV (с содержимым) оттуда же. Для его работы требуется виртуальная машина Java версии не ниже 1.4.

% Для редактирования кода рекомендуется иметь установленным специализированный редактор с подсветкой синтаксиса языка ассемблера, например Geany или jEdit (оба находятся в частности в репозиториях Ubuntu Linux).

% В отдельных случаях возможна работа на компьютерах под управлением операционной системы семейства Windows, однако здесь требуется дополнительная настройка утилит ассемблирования, описанная в методических указаниях~[1] п.~\ref{author-res}.


% 	\ssect[Технические и электронные средства]

% Учёт активности студентов на курсе и основные материалы размещены в системе Moodle, развёрнутой в сети университета по адресу \url{http://edu.mmcs.sfedu.ru}. В дисплейных классах требуется доступ к этому ресурсу посредством браузера актуальной версии. Для проведения контрольных работ необходимо одновременно с доступом к указанному ресурсу ограничить доступ к другим интернет- и интранет-ресурсам, что на данный момент реализовано в лаборатории ММПТ и лаборатории кафедр алгебры и дискретной математики и информатики и вычислительного эксперимента факультета.