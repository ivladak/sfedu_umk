Дисциплина \thecourse{} является частью базовой части профессионального цикла дисциплин.
Необходимыми условиями для освоения дисциплины являются:
\begin{itemize}
\item знания:
\begin{itemize}
\item знание базовых приемов, используемых при проектировании алгоритмов и структур данных,\item воспроизведение базовых алгоритмов, основанных на использовании основных структур данных,\item понимание базовых структур данных и операций над ними,\item технический английский,
\end{itemize}
\item умения:
\begin{itemize}
\item использование окружения программирования,
\item кодирование алгоритмов на одном из языков программирования,
\end{itemize}
\item навыки:
\begin{itemize}
\item алгоритмическое мышление.
\end{itemize}
\end{itemize}


Содержание дисциплины является логическим продолжением содержания дисциплин: 
\begin{itemize}
	\item 
	%Б.3.1.10 
	«Языки программирования»
	\item 
	% Б.3.1.2 
	«Дискретная математика»
	\item 
	% Б.3.1.3 
	«Алгоритмы и структуры данных»
	\item 
	% Б.3.1.4 
	«Теория формальных языков»
	\item 
	% Б.3.1.5 
	«Методы трансляции»
	\item 
	% Б.3.2.2.5 
	«Практикум на ЭВМ»
\end{itemize} 
и служит основой для освоения дисциплин: 
\begin{itemize}
	\item 
	% Б.2.2.1.8 
	«Теория игр и исследования операций»
	\item 
	% Б.3.1.13 
	«Методы оптимизации»
\end{itemize}

% \newpage
% В таблице приведены предшествующие и последующие дисциплины, направленные на формирование компетенций, заявленных в разделе «Цели освоения дисциплины»:

% \begin{longtable}{|c|p{0.15\textwidth}|p{0.35\textwidth}|p{0.35\textwidth}|}\hline
% № п/п &
% \multicolumn{1}{c|}{\pb{Наименование\\компетенции}} &
% \multicolumn{1}{c|}{Предшествующие дисциплины} &
% \multicolumn{1}{c|}{\pb{Последующие дисциплины\\(группы дисциплин)}}\\\hline
% \multicolumn{4}{|l|}{\textit{Общекультурные компетенции}}\\\hline 1 & ОК.14 & Б.2.1.1 «Математический анализ», Б.2.1.2 «Алгебра и геометрия», Б.2.1.3 «Физика», Б.2.2.1.4 «Математическая физика», Б.2.2.1.5 «Функциональный анализ», Б.2.2.1.6 «Концепции современного естествознания», Б.2.2.1.7 «Численные методы», Б.3.1.1 «Безопасность жизнедеятельности», Б.3.1.10 «Языки программирования», Б.3.1.11 «Операционные системы», Б.3.1.2 «Дискретная математика», Б.3.1.3 «Алгоритмы и структуры данных», Б.3.1.4 «Теория формальных языков», Б.3.1.5 «Методы трансляции», Б.3.1.6 «Теория вероятностей и математическая статистика», Б.3.1.8 «Введение в программирование и ЭВМ», Б.3.1.9 «Технологии программирования», Б.3.2.1.1 «Автоматное программирование», Б.3.2.1.2 «Вычислительная геометрия», Б.3.2.1.3 «Параллельное программирование», Б.3.2.1.4 «Теория вычислительной сложности», Б.3.2.2.1 «Парадигмы программирования», Б.3.2.2.1 «Язык программирования Java», Б.3.2.2.2 «Алгоритмы в математике», Б.3.2.2.2 «Специальный семинар», Б.3.2.2.5 «Практикум на ЭВМ», Б.3.2.2.5 «Специальный семинар», Б.5.1 «Производственная практика» & Б.2.2.1.8 «Теория игр и исследования операций», Б.3.1.13 «Методы оптимизации», Б.5.2 «Преддипломная практика»\\\hline
% 2 & ОК.15 & Б.2.1.1 «Математический анализ», Б.2.1.2 «Алгебра и геометрия», Б.2.1.3 «Физика», Б.2.2.1.4 «Математическая физика», Б.2.2.1.5 «Функциональный анализ», Б.2.2.1.6 «Концепции современного естествознания», Б.2.2.1.7 «Численные методы», Б.3.1.1 «Безопасность жизнедеятельности», Б.3.1.10 «Языки программирования», Б.3.1.11 «Операционные системы», Б.3.1.2 «Дискретная математика», Б.3.1.3 «Алгоритмы и структуры данных», Б.3.1.4 «Теория формальных языков», Б.3.1.5 «Методы трансляции», Б.3.1.6 «Теория вероятностей и математическая статистика», Б.3.1.8 «Введение в программирование и ЭВМ», Б.3.1.9 «Технологии программирования», Б.3.2.1.1 «Автоматное программирование», Б.3.2.1.2 «Вычислительная геометрия», Б.3.2.1.3 «Параллельное программирование», Б.3.2.1.4 «Теория вычислительной сложности», Б.3.2.2.1 «Парадигмы программирования», Б.3.2.2.1 «Язык программирования Java», Б.3.2.2.2 «Алгоритмы в математике», Б.3.2.2.2 «Специальный семинар», Б.3.2.2.5 «Практикум на ЭВМ», Б.3.2.2.5 «Специальный семинар», Б.5.1 «Производственная практика» & Б.2.2.1.8 «Теория игр и исследования операций», Б.3.1.13 «Методы оптимизации», Б.5.2 «Преддипломная практика»\\\hline
% 3 & ОК.16 & Б.2.1.1 «Математический анализ», Б.2.1.2 «Алгебра и геометрия», Б.2.1.3 «Физика», Б.2.2.1.4 «Математическая физика», Б.2.2.1.5 «Функциональный анализ», Б.2.2.1.6 «Концепции современного естествознания», Б.2.2.1.7 «Численные методы», Б.3.1.1 «Безопасность жизнедеятельности», Б.3.1.10 «Языки программирования», Б.3.1.11 «Операционные системы», Б.3.1.2 «Дискретная математика», Б.3.1.3 «Алгоритмы и структуры данных», Б.3.1.4 «Теория формальных языков», Б.3.1.5 «Методы трансляции», Б.3.1.6 «Теория вероятностей и математическая статистика», Б.3.1.8 «Введение в программирование и ЭВМ», Б.3.1.9 «Технологии программирования», Б.3.2.1.1 «Автоматное программирование», Б.3.2.1.2 «Вычислительная геометрия», Б.3.2.1.3 «Параллельное программирование», Б.3.2.1.4 «Теория вычислительной сложности», Б.3.2.2.1 «Парадигмы программирования», Б.3.2.2.1 «Язык программирования Java», Б.3.2.2.2 «Алгоритмы в математике», Б.3.2.2.2 «Специальный семинар», Б.3.2.2.5 «Практикум на ЭВМ», Б.3.2.2.5 «Специальный семинар», Б.5.1 «Производственная практика» & Б.2.2.1.8 «Теория игр и исследования операций», Б.3.1.13 «Методы оптимизации», Б.5.2 «Преддипломная практика»\\\hline
% \multicolumn{4}{|l|}{\textit{Профессиональные компетенции}}\\\hline 1 & ПК.3 & Б.2.1.1 «Математический анализ», Б.2.1.2 «Алгебра и геометрия», Б.2.1.3 «Физика», Б.2.2.1.4 «Математическая физика», Б.2.2.1.5 «Функциональный анализ», Б.2.2.1.6 «Концепции современного естествознания», Б.2.2.1.7 «Численные методы», Б.3.1.1 «Безопасность жизнедеятельности», Б.3.1.10 «Языки программирования», Б.3.1.11 «Операционные системы», Б.3.1.2 «Дискретная математика», Б.3.1.3 «Алгоритмы и структуры данных», Б.3.1.4 «Теория формальных языков», Б.3.1.5 «Методы трансляции», Б.3.1.6 «Теория вероятностей и математическая статистика», Б.3.1.8 «Введение в программирование и ЭВМ», Б.3.1.9 «Технологии программирования», Б.3.2.1.1 «Автоматное программирование», Б.3.2.1.2 «Вычислительная геометрия», Б.3.2.1.3 «Параллельное программирование», Б.3.2.1.4 «Теория вычислительной сложности», Б.3.2.2.1 «Парадигмы программирования», Б.3.2.2.1 «Язык программирования Java», Б.3.2.2.2 «Алгоритмы в математике», Б.3.2.2.2 «Специальный семинар», Б.3.2.2.5 «Практикум на ЭВМ», Б.3.2.2.5 «Специальный семинар», Б.5.1 «Производственная практика» & Б.2.2.1.8 «Теория игр и исследования операций», Б.3.1.13 «Методы оптимизации», Б.5.2 «Преддипломная практика»\\\hline
% 2 & ПК.4 & Б.2.1.1 «Математический анализ», Б.2.1.2 «Алгебра и геометрия», Б.2.1.3 «Физика», Б.2.2.1.4 «Математическая физика», Б.2.2.1.5 «Функциональный анализ», Б.2.2.1.6 «Концепции современного естествознания», Б.2.2.1.7 «Численные методы», Б.3.1.1 «Безопасность жизнедеятельности», Б.3.1.10 «Языки программирования», Б.3.1.11 «Операционные системы», Б.3.1.2 «Дискретная математика», Б.3.1.3 «Алгоритмы и структуры данных», Б.3.1.4 «Теория формальных языков», Б.3.1.5 «Методы трансляции», Б.3.1.6 «Теория вероятностей и математическая статистика», Б.3.1.8 «Введение в программирование и ЭВМ», Б.3.1.9 «Технологии программирования», Б.3.2.1.1 «Автоматное программирование», Б.3.2.1.2 «Вычислительная геометрия», Б.3.2.1.3 «Параллельное программирование», Б.3.2.1.4 «Теория вычислительной сложности», Б.3.2.2.1 «Парадигмы программирования», Б.3.2.2.1 «Язык программирования Java», Б.3.2.2.2 «Алгоритмы в математике», Б.3.2.2.2 «Специальный семинар», Б.3.2.2.5 «Практикум на ЭВМ», Б.3.2.2.5 «Специальный семинар», Б.5.1 «Производственная практика» & Б.2.2.1.8 «Теория игр и исследования операций», Б.3.1.13 «Методы оптимизации», Б.5.2 «Преддипломная практика»\\\hline
% 3 & ПК.5 & Б.2.1.1 «Математический анализ», Б.2.1.2 «Алгебра и геометрия», Б.2.1.3 «Физика», Б.2.2.1.4 «Математическая физика», Б.2.2.1.5 «Функциональный анализ», Б.2.2.1.6 «Концепции современного естествознания», Б.2.2.1.7 «Численные методы», Б.3.1.1 «Безопасность жизнедеятельности», Б.3.1.10 «Языки программирования», Б.3.1.11 «Операционные системы», Б.3.1.2 «Дискретная математика», Б.3.1.3 «Алгоритмы и структуры данных», Б.3.1.4 «Теория формальных языков», Б.3.1.5 «Методы трансляции», Б.3.1.6 «Теория вероятностей и математическая статистика», Б.3.1.8 «Введение в программирование и ЭВМ», Б.3.1.9 «Технологии программирования», Б.3.2.1.1 «Автоматное программирование», Б.3.2.1.2 «Вычислительная геометрия», Б.3.2.1.3 «Параллельное программирование», Б.3.2.1.4 «Теория вычислительной сложности», Б.3.2.2.1 «Парадигмы программирования», Б.3.2.2.1 «Язык программирования Java», Б.3.2.2.2 «Алгоритмы в математике», Б.3.2.2.2 «Специальный семинар», Б.3.2.2.5 «Практикум на ЭВМ», Б.3.2.2.5 «Специальный семинар», Б.5.1 «Производственная практика» & Б.2.2.1.8 «Теория игр и исследования операций», Б.3.1.13 «Методы оптимизации», Б.5.2 «Преддипломная практика»\\\hline
% 4 & ПК.7 & Б.2.1.1 «Математический анализ», Б.2.1.2 «Алгебра и геометрия», Б.2.1.3 «Физика», Б.2.2.1.4 «Математическая физика», Б.2.2.1.5 «Функциональный анализ», Б.2.2.1.6 «Концепции современного естествознания», Б.2.2.1.7 «Численные методы», Б.3.1.1 «Безопасность жизнедеятельности», Б.3.1.10 «Языки программирования», Б.3.1.11 «Операционные системы», Б.3.1.2 «Дискретная математика», Б.3.1.3 «Алгоритмы и структуры данных», Б.3.1.4 «Теория формальных языков», Б.3.1.5 «Методы трансляции», Б.3.1.6 «Теория вероятностей и математическая статистика», Б.3.1.8 «Введение в программирование и ЭВМ», Б.3.1.9 «Технологии программирования», Б.3.2.1.1 «Автоматное программирование», Б.3.2.1.2 «Вычислительная геометрия», Б.3.2.1.3 «Параллельное программирование», Б.3.2.1.4 «Теория вычислительной сложности», Б.3.2.2.1 «Парадигмы программирования», Б.3.2.2.1 «Язык программирования Java», Б.3.2.2.2 «Алгоритмы в математике», Б.3.2.2.2 «Специальный семинар», Б.3.2.2.5 «Практикум на ЭВМ», Б.3.2.2.5 «Специальный семинар», Б.5.1 «Производственная практика» & Б.2.2.1.8 «Теория игр и исследования операций», Б.3.1.13 «Методы оптимизации», Б.5.2 «Преддипломная практика»\\\hline

% \end{longtable}


% % использовать \ssect[Абв] или \ssect для нумерации подразделов
% % с факультативным заголовком

% 	\ssect Учебная дисциплина \thecourse{}
% (\theyearofstudy~курс, \theglobalterm~семестр) относится к \ulinepad{
% % математическому и естественнонаучному%
% % /
% профессиональному%
% % обычно видно по учебному плану:
% % м. и ес. там обозначен Б2, п. -- Б3
% % учебные планы ЮФУ: http://sfedu.ru/www/view_plans.startup
% } циклу.

% 	\ssect % пререквизиты, например:
% Для изучения курса \thecourse{}
% студенту достаточно владеть навыками программирования на одном из императивных языков, например, Pascal или C. К особо важным темам базовых курсов по программированию, понимание которых используется в данном курсе, следует отнести следующие:
% \begin{itemize}
% 	\item указатели и прямая работа с памятью,
% 	\item организация типа данных <<массив>>,
% 	\item устройство структуры данных <<линейный односвязный список>>.
% \end{itemize}

% 	\ssect
% В дальнейшем материал данного курса может использоваться в ряде курсов,
% изучаемых на 3--4 курсах, в том числе: компьютерные сети, операционные системы, теория автоматов и формальных языков, параллельное и многопоточное
% программирование.