% Общая трудоемкость дисциплины составляет 4 зачетных единицы, 144 часа.
Общая трудоемкость дисциплины составляет 5 зачетных единиц, 180 часов.

\begin{center}
\begin{longtable}{|c|c|p{0.25\textwidth}|p{1.4cm}|p{1.4cm}|p{1.4cm}|p{1.4cm}|p{1.4cm}|}\hline
\multicolumn{1}{|c|}{\multirow{2}{*}{\parbox[c]{1.7cm}{\bfseriesсеместр}}} &
\multicolumn{1}{c|}{\multirow{2}{*}{\parbox[c]{1.6cm}{\bfseries~\\№\\раздела}}} &
\multicolumn{1}{c|}{\multirow{2}{*}{\parbox[c]{0.2\textwidth}{\bfseries~\\Наименование\\раздела\\дисциплины}}} &
\multicolumn{5}{c|}{\parbox[c]{0.4\textwidth}{\bfseries{}Виды учебной нагрузки и их\\ трудоемкость, часы}}\\\cline{4-8}
& & &
\multicolumn{1}{c|}{\bfseries\begin{sideways}Лекции\end{sideways}} &
\multicolumn{1}{c|}{\bfseries\begin{sideways}\parbox[c]{1.4cm}{Лабораторные~~\\работы}\end{sideways}} &
\multicolumn{1}{c|}{\bfseries\begin{sideways}СРС\end{sideways}} &
\multicolumn{1}{c|}{\bfseries\begin{sideways}\parbox[c]{0.25\textwidth}{Экзамен}\end{sideways}} &
\multicolumn{1}{c|}{\bfseries\begin{sideways}Всего часов\end{sideways}}\\\hline
7 & 1 & Понятие о функциональном программировании & 20 & 20 & 38 & & 78\\\hline
7 & 2 & Интерпретация и компиляция функциональных программ & 16 & 18 & 32 & & 66\\\hline
7 & 1, 2 & Промежуточная аттестация (экзамен) &  &  &  & 36 & \\\hline

\end{longtable}
\end{center}


% \begin{description}
% \item[Раздел 1.] «Понятие о функциональном программировании».\item[Раздел 2.] «Интерпретация и компиляция функциональных программ».
% \end{description}

\ssect{Лекции}

\begin{center}
\begin{longtable}{|c|c|c|p{0.3\textwidth}|c|}\hline
\multicolumn{1}{|c|}{\parbox{1cm}{\bfseries~\\№\\п/п\\~}} &
\multicolumn{1}{c|}{\parbox{2cm}{\bfseries №\\раздела}} &
\multicolumn{1}{c|}{\parbox{3cm}{\bfseries Объем,\\часов}} &
\multicolumn{1}{c|}{\bfseries Тема лекции} &
\multicolumn{1}{c|}{\parbox{3.5cm}{\bfseries Формируемые компетенции}} \\\hline
1 & 1 & 5 & Функциональное программирование & \compone{} \\\hline
2 & 1 & 5 & Введение в язык Haskell & \compone{} \\\hline
3 & 1 & 5 & Функции высших порядков & \compthree{} \\\hline
4 & 1 & 5 & Лямбда-исчисление & \compthree{} \\\hline
5 & 2 & 8 & Представление функциональных программ & \comptwo{} \\\hline
6 & 2 & 8 & Интерпретация & \comptwo{} \\\hline

\multicolumn{3}{|c|}{Итого:} & 36 & \\\hline
\end{longtable}
\end{center}


\ssect{Лабораторные работы}

\begin{center}
\begin{longtable}{|c|c|c|p{0.3\textwidth}|p{0.2\textwidth}|c|}\hline
\multicolumn{1}{|c|}{\parbox[c]{0.6cm}{\bfseries~\\№\\п/п\\~}} &
\multicolumn{1}{c|}{\parbox[c]{1.6cm}{\bfseries №\\раздела}} &
\multicolumn{1}{c|}{\parbox[c]{1.8cm}{\bfseries Трудо-\\ёмкость,\\часов}} &
\multicolumn{1}{c|}{\parbox[c]{3.5cm}{\bfseries Наименование лабораторной\\работы}} &
\multicolumn{1}{c|}{\parbox[c]{3cm}{\bfseries Наименование\\лаборатории}} &
\multicolumn{1}{c|}{\parbox{3.1cm}{\bfseries Формируемые компетенции}} \\\hline
1 & 1 & 5 & Функциональное программирование: проба пера & Компьютерный класс & \compone{} \\\hline
2 & 1 & 5 & Рекурсия & Компьютерный класс & \compone{} \\\hline
3 & 1 & 5 & Работа со строками & Компьютерный класс & \compthree{} \\\hline
4 & 1 & 5 & Дерево & Компьютерный класс & \compthree{} \\\hline
5 & 2 & 9 & Граф & Компьютерный класс & \comptwo{} \\\hline
6 & 2 & 9 & Лямбда-исчисление & Компьютерный класс & \comptwo{} \\\hline

\multicolumn{3}{|c|}{Итого:} & 38 & & \\\hline
\end{longtable}
\end{center}


\ssect{Самостоятельная работа студента}\\

\begin{center}
\begin{longtable}{|c|c|c|p{0.6\textwidth}|}\hline
\multicolumn{1}{|c|}{\parbox[c]{.6cm}{\bfseries~\\№\\п/п\\~}} &
\multicolumn{1}{c|}{\parbox[c]{1.7cm}{\bfseries №\\раздела}} &
\multicolumn{1}{c|}{\parbox[c]{3.1cm}{\bfseries Трудоёмкость,\\часов}} &
\multicolumn{1}{c|}{\parbox[c]{4cm}{\bfseries Вид СРС}} \\\hline
1 & 1 & 19 & Подготовка к лекциям\\\hline
2 & 1 & 19 & Выполнение лабораторных работ\\\hline
3 & 2 & 16 & Подготовка к лекциям\\\hline
4 & 2 & 16 & Выполнение лабораторных работ\\\hline

\multicolumn{2}{|c|}{Итого:} & 70 & \\\hline
\end{longtable}
\end{center}


\ssect{Домашние задания, типовые расчеты и т.п.}\\
Не предусмотрены.

\ssect{Рефераты}\\
Не предусмотрены.

\ssect{Курсовые работы по дисциплине}\\
Не предусмотрены.

\ssect{Виды и содержание учебных занятий}

\textbf{Раздел 1. «Понятие о функциональном программировании»}

{\parindent0pt
\setdescription{leftmargin=\parindent,labelindent=1cm}
\setitemize[1]{leftmargin=1.5cm}

\textbf{Теоретические занятия (лекции)~— 20 часов.}
\begin{description}
\item[Лекция 1.] «Функциональное программирование». Информационная лекция. Рассматриваются следующие вопросы: \begin{itemize}
\item Императивные и функциональные языки\item Функциональный стиль
\end{itemize}\item[Лекция 2.] «Введение в язык Haskell». Информационная лекция. Рассматриваются следующие вопросы: \begin{itemize}
\item История\item Типы данных\item Конструкторы\item Функции\item Рекурсия
\end{itemize}\item[Лекция 3.] «Функции высших порядков». Информационная лекция. Рассматриваются следующие вопросы: \begin{itemize}
\item Определение\item Примеры
\end{itemize}\item[Лекция 4.] «Лямбда-исчисление». Информационная лекция. Рассматриваются следующие вопросы: \begin{itemize}
\item Основы лямбда-исчисления\item Рекурсия в лямбде\item Чистое лямбда-исчисление
\end{itemize}
\end{description}




\textbf{Лабораторный практикум~— 20 часов, 4 работы.}
\begin{description}
\item[Лабораторная работа 1.] «Функциональное программирование: проба пера». Выполняется индивидуально в лаборатории «Компьютерный класс». Задание: \begin{itemize}
\item Студенты должны Написать программу для вычисления приближенного значения числа e по формуле для разложения $e^x$ в ряд Тейлора.\item Предложить программы на языке Паскаль, написанные в традиционном императивном и функциональном стилях.
% \item Студенты должны понимать особенности функционального стиля программирования\item Студенты должны уметь писать программы в функциональном стиле на традиционных языках программирования
\end{itemize}\item[Лабораторная работа 2.] «Рекурсия». Выполняется индивидуально в лаборатории «Компьютерный класс». Задание: \begin{itemize}
\item Написать функцию, определяющую количество строк в списке, содержащих хотя бы одну букву (буквой будем называть символ, для которого заданнаяк функция выдает значение True).
% \item Студенты должны знать основные конструкции языка Haskell\item Студенты должны уметь писать программы на Haskell небольшого размера
\end{itemize}\item[Лабораторная работа 3.] «Работа со строками». Выполняется индивидуально в лаборатории «Компьютерный класс». Задание: \begin{itemize}
\item Написать функцию, вычисляющую длину самой длинной строки в заданном списке строк.
% \item Студенты должны быть знакомы с функциями высших порядков\item Студенты должны уметь реализовывать функции, принимаюищие другие функции в качестве аргумента\item Студенты должны уметь применять на практике функции высших порядков, такие как свертка и map
\end{itemize}\item[Лабораторная работа 4.] «Дерево». Выполняется индивидуально в лаборатории «Компьютерный класс». Задание: \begin{itemize}
\item Дерево задано с помощью следующего описания структуры данных. data Tree a = Node a [Tree a]. То есть дерево представляет собой корневой узел, содержащий некоторое значение произвольного типа a и список поддеревьев. Написать функцию, вычисляющую высоту дерева.
% \item Студенты должны понимать основы лямбда-исчисления\item Студенты должны уметь строить несложные интерфейсы\item Студенты должны уметь писать программы, состоящие из нескольких модулей, связанных интерфейсами
\end{itemize}
\end{description}


\textbf{Раздел 2. «Интерпретация и компиляция функциональных программ»}

{\parindent0pt
\setdescription{leftmargin=\parindent,labelindent=1cm}
\setitemize[1]{leftmargin=1.5cm}

\textbf{Теоретические занятия (лекции)~— 16 часов.}
\begin{description}
\item[Лекция 5.] «Представление функциональных программ». Информационная лекция. Рассматриваются следующие вопросы: \begin{itemize}
\item Компиляция с языка Haskell в расширенное лямбда-исчисление\item Компиляция case-выражения\item Компиляция сопоставления с образцом и связывания\item Представление программ расширенного лямбда-исчисления
\end{itemize}\item[Лекция 6.] «Интерпретация». Информационная лекция. Рассматриваются следующие вопросы: \begin{itemize}
\item Eval/Apply интерпретатор\item Функциональная SECD-машина\item Функциональные эквиваленты императивных программ
\end{itemize}
\end{description}




\textbf{Лабораторный практикум~— 18 часов, 2 работы.}
\begin{description}
\item[Лабораторная работа 5.] «Граф». Выполняется индивидуально в лаборатории «Компьютерный класс». Задание: \begin{itemize}
\item Структура графа задана списками смежности номеров вершин, то есть списком, элементами которого являются пары, состоящие из номера вершины и списка вершин, инцидентных ей\item Написать функцию, которая проверяет, существует ли в графе путь, соединяющий вершины с двумя заданными номерами.
\end{itemize}\item[Лабораторная работа 6.] «Лямбда-исчисление». Выполняется индивидуально в лаборатории «Компьютерный класс». Задание: \begin{itemize}
\item Выполнить редукцию выражения. В нормальном и аппликативном порядке редукций. В обоих случаях найти нормальную форму (НФ) и слабую заголовочную нормальную форму (СЗНФ) выражения.
% \item Студенты должны понимать принципы компиляции и интерпретации программ\item Студенты должны уметь выполнять редукцию лямбда-выражений\item Студенты должны уметь реализовывать структуры данных на функциональных языках программирования\item в рамках которого студенты отвечают на теоретические вопросы, проектируют и реализуют программы с применением навыков и умений,\item полученных в процессе освоения дисциплины.
\end{itemize}
\end{description}
}
