% Общая трудоемкость дисциплины составляет 4 зачетных единицы, 144 часа.
Общая трудоемкость дисциплины составляет 5 зачетных единиц, 180 часов.

\begin{center}
\begin{longtable}{|c|c|p{0.25\textwidth}|p{1.4cm}|p{1.4cm}|p{1.4cm}|p{1.4cm}|p{1.4cm}|}\hline
\multicolumn{1}{|c|}{\multirow{2}{*}{\parbox[c]{1.7cm}{\bfseriesсеместр}}} &
\multicolumn{1}{c|}{\multirow{2}{*}{\parbox[c]{1.6cm}{\bfseries~\\№\\раздела}}} &
\multicolumn{1}{c|}{\multirow{2}{*}{\parbox[c]{0.2\textwidth}{\bfseries~\\Наименование\\раздела\\дисциплины}}} &
\multicolumn{5}{c|}{\parbox[c]{0.4\textwidth}{\bfseries{}Виды учебной нагрузки и их\\ трудоемкость, часы}}\\\cline{4-8}
& & &
\multicolumn{1}{c|}{\bfseries\begin{sideways}Лекции\end{sideways}} &
\multicolumn{1}{c|}{\bfseries\begin{sideways}\parbox[c]{1.4cm}{Лабораторные~~\\работы}\end{sideways}} &
\multicolumn{1}{c|}{\bfseries\begin{sideways}СРС\end{sideways}} &
\multicolumn{1}{c|}{\bfseries\begin{sideways}\parbox[c]{0.25\textwidth}{Экзамен}\end{sideways}} &
\multicolumn{1}{c|}{\bfseries\begin{sideways}Всего часов\end{sideways}}\\\hline
7 & 1 & Понятие о функциональном программировании & 20 & 20 & 38 & & 78\\\hline
7 & 2 & Интерпретация и компиляция функциональных программ & 16 & 18 & 32 & & 66\\\hline
7 & 1, 2 & Промежуточная аттестация (экзамен) &  &  &  & 36 & \\\hline

\end{longtable}
\end{center}


% \begin{description}
% \item[Раздел 1.] «Понятие о функциональном программировании».\item[Раздел 2.] «Интерпретация и компиляция функциональных программ».
% \end{description}

\ssect{Лекции}

\begin{center}
\begin{longtable}{|c|c|c|p{0.3\textwidth}|c|}\hline
\multicolumn{1}{|c|}{\parbox{1cm}{\bfseries~\\№\\п/п\\~}} &
\multicolumn{1}{c|}{\parbox{2cm}{\bfseries №\\раздела}} &
\multicolumn{1}{c|}{\parbox{3cm}{\bfseries Объем,\\часов}} &
\multicolumn{1}{c|}{\bfseries Тема лекции} &
\multicolumn{1}{c|}{\parbox{3.5cm}{\bfseries Формируемые компетенции}} \\\hline
1 & 1 & 5 & Функциональное программирование & \compone{} \\\hline
2 & 1 & 5 & Введение в язык Haskell & \compone{} \\\hline
3 & 1 & 5 & Функции высших порядков & \compthree{} \\\hline
4 & 1 & 5 & Лямбда-исчисление & \compthree{} \\\hline
5 & 2 & 8 & Представление функциональных программ & \comptwo{} \\\hline
6 & 2 & 8 & Интерпретация & \comptwo{} \\\hline

\multicolumn{3}{|c|}{Итого:} & 36 & \\\hline
\end{longtable}
\end{center}


\ssect{Лабораторные работы}

\begin{center}
\begin{longtable}{|c|c|c|p{0.3\textwidth}|p{0.2\textwidth}|c|}\hline
\multicolumn{1}{|c|}{\parbox[c]{0.6cm}{\bfseries~\\№\\п/п\\~}} &
\multicolumn{1}{c|}{\parbox[c]{1.6cm}{\bfseries №\\раздела}} &
\multicolumn{1}{c|}{\parbox[c]{1.8cm}{\bfseries Трудо-\\ёмкость,\\часов}} &
\multicolumn{1}{c|}{\parbox[c]{3.5cm}{\bfseries Наименование лабораторной\\работы}} &
\multicolumn{1}{c|}{\parbox[c]{3cm}{\bfseries Наименование\\лаборатории}} &
\multicolumn{1}{c|}{\parbox{3.1cm}{\bfseries Формируемые компетенции}} \\\hline
1 & 1 & 5 & Функциональное программирование: проба пера & Компьютерный класс & \compone{} \\\hline
2 & 1 & 5 & Рекурсия & Компьютерный класс & \compone{} \\\hline
3 & 1 & 5 & Работа со строками & Компьютерный класс & \compthree{} \\\hline
4 & 1 & 5 & Дерево & Компьютерный класс & \compthree{} \\\hline
5 & 2 & 9 & Граф & Компьютерный класс & \comptwo{} \\\hline
6 & 2 & 9 & Лямбда-исчисление & Компьютерный класс & \comptwo{} \\\hline

\multicolumn{3}{|c|}{Итого:} & 38 & & \\\hline
\end{longtable}
\end{center}


\ssect{Самостоятельная работа студента}\\

\begin{center}
\begin{longtable}{|c|c|c|p{0.6\textwidth}|}\hline
\multicolumn{1}{|c|}{\parbox[c]{.6cm}{\bfseries~\\№\\п/п\\~}} &
\multicolumn{1}{c|}{\parbox[c]{1.7cm}{\bfseries №\\раздела}} &
\multicolumn{1}{c|}{\parbox[c]{3.1cm}{\bfseries Трудоёмкость,\\часов}} &
\multicolumn{1}{c|}{\parbox[c]{4cm}{\bfseries Вид СРС}} \\\hline
1 & 1 & 19 & Подготовка к лекциям\\\hline
2 & 1 & 19 & Выполнение лабораторных работ\\\hline
3 & 2 & 16 & Подготовка к лекциям\\\hline
4 & 2 & 16 & Выполнение лабораторных работ\\\hline

\multicolumn{2}{|c|}{Итого:} & 70 & \\\hline
\end{longtable}
\end{center}


\ssect{Домашние задания, типовые расчеты и т.п.}\\
Не предусмотрены.

\ssect{Рефераты}\\
Не предусмотрены.

\ssect{Курсовые работы по дисциплине}\\
Не предусмотрены.