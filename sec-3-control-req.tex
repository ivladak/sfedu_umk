% В разделе 3 РПД устанавливаются требования к уровню освоения дисциплины в соответствии с компетентностной моделью выпускника, изложенной в ООП направления подготовки (специальности). Должны быть указаны коды и содержание компетенций, отнесенных к дисциплине в «Матрице соответствия компетенций и составных частей ООП», содержащейся в ООП направления подготовки (специальности). По каждой компетенции, в формировании которой участвует дисциплина, должны быть сформулированы требования к студенту, освоившему программу дисциплины, по схеме: иметь представление – знать – уметь – иметь опыт

% 	\ssect
Процесс изучения дисциплины направлен на формирование элементов следующих компетенций в соответствии с ФГОС ВПО и ООП ВПО по данному направлению подготовки:
% % общий перечень компетенций по направлению подготовки обычно
% % можно найти в госстандарте или в ООП вуза. ФГОСы размещены на одном из
% % специализированных сайтов Минобра. Сейчас это:
% % http://fgosvo.ru/fgosvpo/7/6/1/28
% % ООП направлений ЮФУ должны быть на офсайте
% % на данный момент (середина 2014) это:
% % http://sfedu.ru/www/edu.NaprPodg_show?v_snp_date_in=01.01.2009
% % существующий на данный момент перечень для ПМИ и ФИИТ вынесен в файл:
% % https://docs.google.com/document/d/12WMvvjyVEkF9S5gQVCI26zMnVXGQJcVEv18Qjz_x9yk/edit?usp=sharing
\begin{enumerate}
% \rusitems % нумерация кириллическими буквами
	\item универсальных (УК):
	\begin{itemize}
		\item иметь готовность использовать современные методы и технологии научной коммуникации на государственном и иностранном языках (\compone{})
	\end{itemize}

	\item профессиональных (ПК):
	\begin{itemize}
		\item иметь способность в составе научно-исследовательского и производственного коллектива решать задачи профессиональной деятельности (\comptwo{});
		\item обладать способностью критически переосмысливать накопленный опыт, изменять при необходимости вид и характер своей профессиональной деятельности (\compthree{}).
	\end{itemize}
\end{enumerate}

В результате освоения дисциплины обучающийся должен

% \noindent\textbf{знать:}
% 	\begin{itemize}[topsep=1mm]

% 	\end{itemize}

\noindent\textbf{уметь:}
	\begin{itemize}[topsep=1mm]
		\item писать программы в функциональном стиле на традиционных языках программирования
		\item писать программы на Haskell небольшого размера
		\item реализовывать функции, принимаюищие другие функции в качестве аргумента
		\item применять на практике функции высших порядков, такие как свертка и \texttt{map}
		\item строить несложные интерфейсы
		\item писать программы, состоящие из нескольких модулей, связанных интерфейсами
		\item выполнять редукцию лямбда-выражений
		\item реализовывать структуры данных на функциональных языках программирования
	\end{itemize}

% \noindent\textbf{владеть:}
% 	\begin{itemize}[topsep=1mm]
% 	\end{itemize}