\textbf{Текущая аттестация} студентов производится лектором и преподавателем (преподавателями), ведущими лабораторные работы и практические занятия по дисциплине в следующих формах:
\begin{itemize}


\item выполнение лабораторных работ;
\item защита лабораторных работ;
\item отдельно оцениваются личностные качества студента.
\end{itemize}

\textbf{Рубежная аттестация} студентов производится по окончании модуля в следующих формах:
\begin{itemize}
\item защита лабораторных работ.
\end{itemize}

\textbf{Промежуточный контроль} по результатам семестра по дисциплине проходит:
\begin{itemize}
\item в форме устного экзамена.
\end{itemize}

Фонды оценочных средств и критерии оценивания приведены в приложении к Рабочей программе.
% 	\ssect
% Процесс изучения дисциплины направлен на формирование элементов следующих компетенций в соответствии с ФГОС ВПО (ОС ЮФУ) и ООП ВПО по данному направлению подготовки:
% % общий перечень компетенций по направлению подготовки обычно
% % можно найти в госстандарте или в ООП вуза. ФГОСы размещены на одном из
% % специализированных сайтов Минобра. Сейчас это:
% % http://fgosvo.ru/fgosvpo/7/6/1/28
% % ООП направлений ЮФУ должны быть на офсайте
% % на данный момент (середина 2014) это:
% % http://sfedu.ru/www/edu.NaprPodg_show?v_snp_date_in=01.01.2009
% % существующий на данный момент перечень для ПМИ и ФИИТ вынесен в файл:
% % https://docs.google.com/document/d/12WMvvjyVEkF9S5gQVCI26zMnVXGQJcVEv18Qjz_x9yk/edit?usp=sharing
% \begin{enumerate}
% \rusitems % нумерация кириллическими буквами
% 	\item общекультурных (ОК):
% 	\begin{itemize}
% 		\item анализировать основные этапы и закономерности исторического развития общества для формирования гражданской позиции (ОК-2);
% 		\item использовать основы экономических знаний в различных сферах жизнедеятельности (ОК-3);
% 	\end{itemize}

% 	\item общепрофессиональных (ОПК):
% 	\begin{itemize}
% 		\item использовать базовые знания естественных наук, математики и информатики, основные факты, концепции, принципы теорий, связанных с фундаментальной информатикой и информационными технологиями (ОПК-1);
% 	\end{itemize}

% 	\item профессиональных (ПК):
% 	\begin{itemize}
% 		\item понимать, совершенствовать и применять современный математический аппарат, фундаментальные концепции и системные методологии, международные и профессиональные стандарты в области информационных технологий (ПК-2);
% 		\item критически переосмысливать накопленный опыт, изменять при необходимости вид и характер своей профессиональной деятельности (ПК-5).
% 	\end{itemize}
% \end{enumerate}

% В результате освоения дисциплины обучающийся должен

% \noindent\textbf{знать:}
% 	\begin{itemize}[topsep=1mm]
% 		\item основные этапы развития вычислительной техники,
% 		\item примеры применения компьютеров в современном обществе,
% 		\item наиболее широкоупотребительные способы классификации компьютеров,
% 		\item составляющие части вычислительной системы и проблематику их разработки и взаимодействия,
% 		\item уровни архитектуры современной вычислительной машины, их назначение и взаимодействие,
% 		\item основные цифровые логические схемы и шины, используемые в компьютерах,
% 		\item проблематику микропрограммного управления центральным процессором,
% 		\item задачи, решаемые уровнем набора инструкций,
% 		\item способы представления целых знаковых чисел в компьютере,
% 		\item способы представления рациональных чисел в компьютере: числа с фиксированной точкой, числа с плавающей точкой, стандарт IEEE~754,
% 		\item ядро набора инструкций x86;
% 	\end{itemize}

% \noindent\textbf{уметь:}
% 	\begin{itemize}[topsep=1mm]
% 		\item различать вычислительные машины, относящиеся к разным поколениям,
% 		\item определять тип вычислительной архитектуры в каждой из основных классификаций: фоннеймановская/гарвардская, CISC/RISC, SISD/SIMD/MISD/MIMD,
% 		\item определять представление целых беззнаковых чисел в машинах с различной организацией ОЗУ: разным размером байт и разным порядком байт в слове,
% 		\item использовать простейшие помехоустойчивые коды: проверки чётности, (7,4)"/Хэм\-мин\-га,
% 		\item прогнозировать поведение простейших цифровых логических схем,
% 		\item составлять фрагменты микропрограммы для простейших микроархитектур,
% 		\item определять представление целых знаковых чисел и чисел с плавающей точкой по стандарту IEEE~754 в памяти компьютера,
% 		\item составлять простейшие программы на языке ассемблера для процессора семейства x86;
% 	\end{itemize}

% \noindent\textbf{владеть:}
% 	\begin{itemize}[topsep=1mm]
% 		\item терминологией из области архитектуры компьютера,
% 		\item средствами ассемблирования программ.
% 	\end{itemize}