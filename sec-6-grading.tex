%Оценка качества освоения дисциплины студентами

Оценивание уровня учебных достижений студента осуществляется в виде текущей аттестации, рубежной аттестации и промежуточного контроля.

\textbf{Текущая аттестация} студентов производится лектором и преподавателем (преподавателями), ведущими лабораторные работы и практические занятия по дисциплине в следующих формах:
\begin{itemize}

\item выполнение лабораторных работ;
\item защита лабораторных работ;
\item отдельно оцениваются личностные качества студента.
\end{itemize}

\textbf{Рубежная аттестация} студентов производится в следующих формах:
\begin{itemize}
\item защита лабораторных работ.
\end{itemize}

\textbf{Промежуточный контроль} по результатам семестра по дисциплине проходит:
\begin{itemize}
\item в форме устного экзамена.
\end{itemize}


\subsection*{Фонды оценочных средств}

Фонды оценочных средств, позволяющие оценить РО по данной дисциплине, включают в себя:
\begin{itemize}
\item шаблоны отчётов по лабораторным работам, выдаются индивидуально.
\end{itemize}

\subsection*{Критерии оценивания}


\noindent\textbf{Лабораторные работы}\\
\textit{Допуск за защите ЛР}\\
Допуск к защите ЛР происходит в форме устного тестирования направленного на проверку самостоятельности выполнения ЛР. При ответе на более чем 60\% вопросов студент допускается к защите ЛР.\\

\noindent\textit{Защита ЛР}\\
Отчет по лабораторной работе представляется в электронном виде в формате, предусмотренном шаблоном отчета по лабораторной работе.
Защита отчета проходит в форме доклада студента по выполненной работе и ответов на вопросы преподавателя.
В случае если содержание и оформление отчета, а также поведение студента во время защиты соответствуют указанным требованиям, студент получает максимальное количество баллов.

Основаниями для снижения количества баллов в диапазоне от max до min являются:
\begin{itemize}
\item небрежное выполнение,
\item низкое качество графического материала.
\end{itemize}

Отчет не может быть принят и подлежит доработке в случае:
\begin{itemize}
\item отсутствия необходимых разделов,
\item отсутствия необходимого графического материала,
\item некорректной обработки результатов.
\end{itemize}
