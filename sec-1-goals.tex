Целью освоения дисциплины является достижение следующих результатов образования (РО):\\
{\parindent0pt

\textbf{знания}:
\begin{itemize}
\item на уровне представлений:
\begin{itemize}
\item представление о принципах построения программ высокого качества на функциональных языках,\item представление о понятии функций высших порядков,\item лямбда-исчисление, интерпретация и компиляция функциональных програм,
\end{itemize}


\item на уровне понимания:
\begin{itemize}
\item понимание общих правил построения хорошего программного кода,
\end{itemize}
\end{itemize}

\textbf{навыки}:
\begin{itemize}
\item навык разработки программ среднего размера, находить и устранять их возможные уязвимости,\item навык написания программ на языке программирования Haskell,\item навык проектирования программ на функциональном языке программирования,\item навык проведения вычислений в лямбда-исчислении,\item навык использования библиотек функциональных языков программирования.
\end{itemize}

Перечисленные РО являются основой для формирования следующих компетенций:
\begin{itemize}
\item универсальных: \begin{itemize}
\item \compone{} — готовностью использовать современные методы и технологии научной коммуникации на государственном и иностранном языках,
% \item ОК.15 — способностью работы с информацией из различных источников, включая сетевые ресурсы сети Интернет, для решения профессиональных и социальных задач,
% \item ОК.16 — способностью к интеллектуальному, культурному, нравственному, физическому и профессиональному саморазвитию, стремление к повышению своей квалификации и мастерства,
\end{itemize}\item профессиональных: \begin{itemize}
% \item ПК.3 — способностью понимать и применять в исследовательской и прикладной деятельности современный математический аппарат,
\item \comptwo{} — способностью в составе научно-исследовательского и производственного коллектива решать задачи профессиональной деятельности,
\item \compthree{} — способностью критически переосмысливать накопленный опыт, изменять при необходимости вид и характер своей профессиональной деятельности,
% \item ПК.7 — способностью собирать, обрабатывать и интерпретировать данные современных научных исследований, необходимые для формирования выводов по соответствующим научным, профессиональным, социальным и этическим проблемам,
\end{itemize}
\end{itemize}
}