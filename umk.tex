\input umk_preamble

\input general_info.tex

\begin{document}

%\tableofcontents

\section{Цели и задачи освоения дисциплины}
Целью освоения дисциплины является достижение следующих результатов образования (РО):\\
{\parindent0pt

\textbf{знания}:
\begin{itemize}
\item на уровне представлений:
\begin{itemize}
\item представление о принципах построения программ высокого качества на функциональных языках,\item представление о понятии функций высших порядков,\item лямбда-исчисление, интерпретация и компиляция функциональных програм,
\end{itemize}


\item на уровне понимания:
\begin{itemize}
\item понимание общих правил построения хорошего программного кода,
\end{itemize}
\end{itemize}

\textbf{навыки}:
\begin{itemize}
\item навык разработки программ среднего размера, находить и устранять их возможные уязвимости,\item навык написания программ на языке программирования Haskell,\item навык проектирования программ на функциональном языке программирования,\item навык проведения вычислений в лямбда-исчислении,\item навык использования библиотек функциональных языков программирования.
\end{itemize}

Перечисленные РО являются основой для формирования следующих компетенций:
\begin{itemize}
\item универсальных: \begin{itemize}
\item \compone{} — готовностью использовать современные методы и технологии научной коммуникации на государственном и иностранном языках,
% \item ОК.15 — способностью работы с информацией из различных источников, включая сетевые ресурсы сети Интернет, для решения профессиональных и социальных задач,
% \item ОК.16 — способностью к интеллектуальному, культурному, нравственному, физическому и профессиональному саморазвитию, стремление к повышению своей квалификации и мастерства,
\end{itemize}\item профессиональных: \begin{itemize}
% \item ПК.3 — способностью понимать и применять в исследовательской и прикладной деятельности современный математический аппарат,
\item \comptwo{} — способностью в составе научно-исследовательского и производственного коллектива решать задачи профессиональной деятельности,
\item \compthree{} — способностью критически переосмысливать накопленный опыт, изменять при необходимости вид и характер своей профессиональной деятельности,
% \item ПК.7 — способностью собирать, обрабатывать и интерпретировать данные современных научных исследований, необходимые для формирования выводов по соответствующим научным, профессиональным, социальным и этическим проблемам,
\end{itemize}
\end{itemize}
}

\section{Место дисциплины в структуре ООП ВПО}
Дисциплина \thecourse{} является частью базовой части профессионального цикла дисциплин.
Необходимыми условиями для освоения дисциплины являются:
\begin{itemize}
\item знания:
\begin{itemize}
\item знание базовых приемов, используемых при проектировании алгоритмов и структур данных,\item воспроизведение базовых алгоритмов, основанных на использовании основных структур данных,\item понимание базовых структур данных и операций над ними,\item технический английский,
\end{itemize}
\item умения:
\begin{itemize}
\item особенности сетевого взаимодействия операционных систем,
\end{itemize}
\item навыки:
\begin{itemize}
\item алгоритмическое мышление.
\end{itemize}
\end{itemize}


Содержание дисциплины является логическим продолжением содержания дисциплин: \begin{itemize}
\item Б.3.1.10 «Языки программирования»\item Б.3.1.2 «Дискретная математика»\item Б.3.1.3 «Алгоритмы и структуры данных»\item Б.3.1.4 «Теория формальных языков»\item Б.3.1.5 «Методы трансляции»\item Б.3.1.8 «Введение в программирование и ЭВМ»\item Б.3.2.2.5 «Практикум на ЭВМ»
\end{itemize} и служит основой для освоения дисциплин: \begin{itemize}
\item Б.2.2.1.8 «Теория игр и исследования операций»\item Б.3.1.13 «Методы оптимизации»
\end{itemize}

% \newpage
% В таблице приведены предшествующие и последующие дисциплины, направленные на формирование компетенций, заявленных в разделе «Цели освоения дисциплины»:

% \begin{longtable}{|c|p{0.15\textwidth}|p{0.35\textwidth}|p{0.35\textwidth}|}\hline
% № п/п &
% \multicolumn{1}{c|}{\pb{Наименование\\компетенции}} &
% \multicolumn{1}{c|}{Предшествующие дисциплины} &
% \multicolumn{1}{c|}{\pb{Последующие дисциплины\\(группы дисциплин)}}\\\hline
% \multicolumn{4}{|l|}{\textit{Общекультурные компетенции}}\\\hline 1 & ОК.14 & Б.2.1.1 «Математический анализ», Б.2.1.2 «Алгебра и геометрия», Б.2.1.3 «Физика», Б.2.2.1.4 «Математическая физика», Б.2.2.1.5 «Функциональный анализ», Б.2.2.1.6 «Концепции современного естествознания», Б.2.2.1.7 «Численные методы», Б.3.1.1 «Безопасность жизнедеятельности», Б.3.1.10 «Языки программирования», Б.3.1.11 «Операционные системы», Б.3.1.2 «Дискретная математика», Б.3.1.3 «Алгоритмы и структуры данных», Б.3.1.4 «Теория формальных языков», Б.3.1.5 «Методы трансляции», Б.3.1.6 «Теория вероятностей и математическая статистика», Б.3.1.8 «Введение в программирование и ЭВМ», Б.3.1.9 «Технологии программирования», Б.3.2.1.1 «Автоматное программирование», Б.3.2.1.2 «Вычислительная геометрия», Б.3.2.1.3 «Параллельное программирование», Б.3.2.1.4 «Теория вычислительной сложности», Б.3.2.2.1 «Парадигмы программирования», Б.3.2.2.1 «Язык программирования Java», Б.3.2.2.2 «Алгоритмы в математике», Б.3.2.2.2 «Специальный семинар», Б.3.2.2.5 «Практикум на ЭВМ», Б.3.2.2.5 «Специальный семинар», Б.5.1 «Производственная практика» & Б.2.2.1.8 «Теория игр и исследования операций», Б.3.1.13 «Методы оптимизации», Б.5.2 «Преддипломная практика»\\\hline
% 2 & ОК.15 & Б.2.1.1 «Математический анализ», Б.2.1.2 «Алгебра и геометрия», Б.2.1.3 «Физика», Б.2.2.1.4 «Математическая физика», Б.2.2.1.5 «Функциональный анализ», Б.2.2.1.6 «Концепции современного естествознания», Б.2.2.1.7 «Численные методы», Б.3.1.1 «Безопасность жизнедеятельности», Б.3.1.10 «Языки программирования», Б.3.1.11 «Операционные системы», Б.3.1.2 «Дискретная математика», Б.3.1.3 «Алгоритмы и структуры данных», Б.3.1.4 «Теория формальных языков», Б.3.1.5 «Методы трансляции», Б.3.1.6 «Теория вероятностей и математическая статистика», Б.3.1.8 «Введение в программирование и ЭВМ», Б.3.1.9 «Технологии программирования», Б.3.2.1.1 «Автоматное программирование», Б.3.2.1.2 «Вычислительная геометрия», Б.3.2.1.3 «Параллельное программирование», Б.3.2.1.4 «Теория вычислительной сложности», Б.3.2.2.1 «Парадигмы программирования», Б.3.2.2.1 «Язык программирования Java», Б.3.2.2.2 «Алгоритмы в математике», Б.3.2.2.2 «Специальный семинар», Б.3.2.2.5 «Практикум на ЭВМ», Б.3.2.2.5 «Специальный семинар», Б.5.1 «Производственная практика» & Б.2.2.1.8 «Теория игр и исследования операций», Б.3.1.13 «Методы оптимизации», Б.5.2 «Преддипломная практика»\\\hline
% 3 & ОК.16 & Б.2.1.1 «Математический анализ», Б.2.1.2 «Алгебра и геометрия», Б.2.1.3 «Физика», Б.2.2.1.4 «Математическая физика», Б.2.2.1.5 «Функциональный анализ», Б.2.2.1.6 «Концепции современного естествознания», Б.2.2.1.7 «Численные методы», Б.3.1.1 «Безопасность жизнедеятельности», Б.3.1.10 «Языки программирования», Б.3.1.11 «Операционные системы», Б.3.1.2 «Дискретная математика», Б.3.1.3 «Алгоритмы и структуры данных», Б.3.1.4 «Теория формальных языков», Б.3.1.5 «Методы трансляции», Б.3.1.6 «Теория вероятностей и математическая статистика», Б.3.1.8 «Введение в программирование и ЭВМ», Б.3.1.9 «Технологии программирования», Б.3.2.1.1 «Автоматное программирование», Б.3.2.1.2 «Вычислительная геометрия», Б.3.2.1.3 «Параллельное программирование», Б.3.2.1.4 «Теория вычислительной сложности», Б.3.2.2.1 «Парадигмы программирования», Б.3.2.2.1 «Язык программирования Java», Б.3.2.2.2 «Алгоритмы в математике», Б.3.2.2.2 «Специальный семинар», Б.3.2.2.5 «Практикум на ЭВМ», Б.3.2.2.5 «Специальный семинар», Б.5.1 «Производственная практика» & Б.2.2.1.8 «Теория игр и исследования операций», Б.3.1.13 «Методы оптимизации», Б.5.2 «Преддипломная практика»\\\hline
% \multicolumn{4}{|l|}{\textit{Профессиональные компетенции}}\\\hline 1 & ПК.3 & Б.2.1.1 «Математический анализ», Б.2.1.2 «Алгебра и геометрия», Б.2.1.3 «Физика», Б.2.2.1.4 «Математическая физика», Б.2.2.1.5 «Функциональный анализ», Б.2.2.1.6 «Концепции современного естествознания», Б.2.2.1.7 «Численные методы», Б.3.1.1 «Безопасность жизнедеятельности», Б.3.1.10 «Языки программирования», Б.3.1.11 «Операционные системы», Б.3.1.2 «Дискретная математика», Б.3.1.3 «Алгоритмы и структуры данных», Б.3.1.4 «Теория формальных языков», Б.3.1.5 «Методы трансляции», Б.3.1.6 «Теория вероятностей и математическая статистика», Б.3.1.8 «Введение в программирование и ЭВМ», Б.3.1.9 «Технологии программирования», Б.3.2.1.1 «Автоматное программирование», Б.3.2.1.2 «Вычислительная геометрия», Б.3.2.1.3 «Параллельное программирование», Б.3.2.1.4 «Теория вычислительной сложности», Б.3.2.2.1 «Парадигмы программирования», Б.3.2.2.1 «Язык программирования Java», Б.3.2.2.2 «Алгоритмы в математике», Б.3.2.2.2 «Специальный семинар», Б.3.2.2.5 «Практикум на ЭВМ», Б.3.2.2.5 «Специальный семинар», Б.5.1 «Производственная практика» & Б.2.2.1.8 «Теория игр и исследования операций», Б.3.1.13 «Методы оптимизации», Б.5.2 «Преддипломная практика»\\\hline
% 2 & ПК.4 & Б.2.1.1 «Математический анализ», Б.2.1.2 «Алгебра и геометрия», Б.2.1.3 «Физика», Б.2.2.1.4 «Математическая физика», Б.2.2.1.5 «Функциональный анализ», Б.2.2.1.6 «Концепции современного естествознания», Б.2.2.1.7 «Численные методы», Б.3.1.1 «Безопасность жизнедеятельности», Б.3.1.10 «Языки программирования», Б.3.1.11 «Операционные системы», Б.3.1.2 «Дискретная математика», Б.3.1.3 «Алгоритмы и структуры данных», Б.3.1.4 «Теория формальных языков», Б.3.1.5 «Методы трансляции», Б.3.1.6 «Теория вероятностей и математическая статистика», Б.3.1.8 «Введение в программирование и ЭВМ», Б.3.1.9 «Технологии программирования», Б.3.2.1.1 «Автоматное программирование», Б.3.2.1.2 «Вычислительная геометрия», Б.3.2.1.3 «Параллельное программирование», Б.3.2.1.4 «Теория вычислительной сложности», Б.3.2.2.1 «Парадигмы программирования», Б.3.2.2.1 «Язык программирования Java», Б.3.2.2.2 «Алгоритмы в математике», Б.3.2.2.2 «Специальный семинар», Б.3.2.2.5 «Практикум на ЭВМ», Б.3.2.2.5 «Специальный семинар», Б.5.1 «Производственная практика» & Б.2.2.1.8 «Теория игр и исследования операций», Б.3.1.13 «Методы оптимизации», Б.5.2 «Преддипломная практика»\\\hline
% 3 & ПК.5 & Б.2.1.1 «Математический анализ», Б.2.1.2 «Алгебра и геометрия», Б.2.1.3 «Физика», Б.2.2.1.4 «Математическая физика», Б.2.2.1.5 «Функциональный анализ», Б.2.2.1.6 «Концепции современного естествознания», Б.2.2.1.7 «Численные методы», Б.3.1.1 «Безопасность жизнедеятельности», Б.3.1.10 «Языки программирования», Б.3.1.11 «Операционные системы», Б.3.1.2 «Дискретная математика», Б.3.1.3 «Алгоритмы и структуры данных», Б.3.1.4 «Теория формальных языков», Б.3.1.5 «Методы трансляции», Б.3.1.6 «Теория вероятностей и математическая статистика», Б.3.1.8 «Введение в программирование и ЭВМ», Б.3.1.9 «Технологии программирования», Б.3.2.1.1 «Автоматное программирование», Б.3.2.1.2 «Вычислительная геометрия», Б.3.2.1.3 «Параллельное программирование», Б.3.2.1.4 «Теория вычислительной сложности», Б.3.2.2.1 «Парадигмы программирования», Б.3.2.2.1 «Язык программирования Java», Б.3.2.2.2 «Алгоритмы в математике», Б.3.2.2.2 «Специальный семинар», Б.3.2.2.5 «Практикум на ЭВМ», Б.3.2.2.5 «Специальный семинар», Б.5.1 «Производственная практика» & Б.2.2.1.8 «Теория игр и исследования операций», Б.3.1.13 «Методы оптимизации», Б.5.2 «Преддипломная практика»\\\hline
% 4 & ПК.7 & Б.2.1.1 «Математический анализ», Б.2.1.2 «Алгебра и геометрия», Б.2.1.3 «Физика», Б.2.2.1.4 «Математическая физика», Б.2.2.1.5 «Функциональный анализ», Б.2.2.1.6 «Концепции современного естествознания», Б.2.2.1.7 «Численные методы», Б.3.1.1 «Безопасность жизнедеятельности», Б.3.1.10 «Языки программирования», Б.3.1.11 «Операционные системы», Б.3.1.2 «Дискретная математика», Б.3.1.3 «Алгоритмы и структуры данных», Б.3.1.4 «Теория формальных языков», Б.3.1.5 «Методы трансляции», Б.3.1.6 «Теория вероятностей и математическая статистика», Б.3.1.8 «Введение в программирование и ЭВМ», Б.3.1.9 «Технологии программирования», Б.3.2.1.1 «Автоматное программирование», Б.3.2.1.2 «Вычислительная геометрия», Б.3.2.1.3 «Параллельное программирование», Б.3.2.1.4 «Теория вычислительной сложности», Б.3.2.2.1 «Парадигмы программирования», Б.3.2.2.1 «Язык программирования Java», Б.3.2.2.2 «Алгоритмы в математике», Б.3.2.2.2 «Специальный семинар», Б.3.2.2.5 «Практикум на ЭВМ», Б.3.2.2.5 «Специальный семинар», Б.5.1 «Производственная практика» & Б.2.2.1.8 «Теория игр и исследования операций», Б.3.1.13 «Методы оптимизации», Б.5.2 «Преддипломная практика»\\\hline

% \end{longtable}


% % использовать \ssect[Абв] или \ssect для нумерации подразделов
% % с факультативным заголовком

% 	\ssect Учебная дисциплина \thecourse{}
% (\theyearofstudy~курс, \theglobalterm~семестр) относится к \ulinepad{
% % математическому и естественнонаучному%
% % /
% профессиональному%
% % обычно видно по учебному плану:
% % м. и ес. там обозначен Б2, п. -- Б3
% % учебные планы ЮФУ: http://sfedu.ru/www/view_plans.startup
% } циклу.

% 	\ssect % пререквизиты, например:
% Для изучения курса \thecourse{}
% студенту достаточно владеть навыками программирования на одном из императивных языков, например, Pascal или C. К особо важным темам базовых курсов по программированию, понимание которых используется в данном курсе, следует отнести следующие:
% \begin{itemize}
% 	\item указатели и прямая работа с памятью,
% 	\item организация типа данных <<массив>>,
% 	\item устройство структуры данных <<линейный односвязный список>>.
% \end{itemize}

% 	\ssect
% В дальнейшем материал данного курса может использоваться в ряде курсов,
% изучаемых на 3--4 курсах, в том числе: компьютерные сети, операционные системы, теория автоматов и формальных языков, параллельное и многопоточное
% программирование.

\section{Требования к результатам освоения содержания дисциплины}
% В разделе 3 РПД устанавливаются требования к уровню освоения дисциплины в соответствии с компетентностной моделью выпускника, изложенной в ООП направления подготовки (специальности). Должны быть указаны коды и содержание компетенций, отнесенных к дисциплине в «Матрице соответствия компетенций и составных частей ООП», содержащейся в ООП направления подготовки (специальности). По каждой компетенции, в формировании которой участвует дисциплина, должны быть сформулированы требования к студенту, освоившему программу дисциплины, по схеме: иметь представление – знать – уметь – иметь опыт

% 	\ssect
Процесс изучения дисциплины направлен на формирование элементов следующих компетенций в соответствии с ФГОС ВПО и ООП ВПО по данному направлению подготовки:
% % общий перечень компетенций по направлению подготовки обычно
% % можно найти в госстандарте или в ООП вуза. ФГОСы размещены на одном из
% % специализированных сайтов Минобра. Сейчас это:
% % http://fgosvo.ru/fgosvpo/7/6/1/28
% % ООП направлений ЮФУ должны быть на офсайте
% % на данный момент (середина 2014) это:
% % http://sfedu.ru/www/edu.NaprPodg_show?v_snp_date_in=01.01.2009
% % существующий на данный момент перечень для ПМИ и ФИИТ вынесен в файл:
% % https://docs.google.com/document/d/12WMvvjyVEkF9S5gQVCI26zMnVXGQJcVEv18Qjz_x9yk/edit?usp=sharing
\begin{enumerate}
% \rusitems % нумерация кириллическими буквами
	\item универсальных (УК):
	\begin{itemize}
		\item иметь готовность использовать современные методы и технологии научной коммуникации на государственном и иностранном языках (\compone{})
	\end{itemize}

	\item профессиональных (ПК):
	\begin{itemize}
		\item иметь способность в составе научно-исследовательского и производственного коллектива решать задачи профессиональной деятельности (\comptwo{});
		\item обладать способностью критически переосмысливать накопленный опыт, изменять при необходимости вид и характер своей профессиональной деятельности (\compthree{}).
	\end{itemize}
\end{enumerate}

В результате освоения дисциплины обучающийся должен

% \noindent\textbf{знать:}
% 	\begin{itemize}[topsep=1mm]

% 	\end{itemize}

\noindent\textbf{уметь:}
	\begin{itemize}[topsep=1mm]
		\item писать программы в функциональном стиле на традиционных языках программирования
		\item писать программы на Haskell небольшого размера
		\item реализовывать функции, принимающие другие функции в качестве аргумента
		\item применять на практике функции высших порядков, такие как свертка и \texttt{map}
		\item строить несложные интерфейсы
		\item писать программы, состоящие из нескольких модулей, связанных интерфейсами
		\item выполнять редукцию лямбда-выражений
		\item реализовывать структуры данных на функциональных языках программирования
	\end{itemize}
 
\noindent\textbf{иметь опыт:}
	\begin{itemize}[topsep=1mm]
		\item чтения формальной документации языков программирования и библиотек на английском языке
		\item разработки программ в функциональном стиле
	\end{itemize}

\noindent\textbf{иметь представление:}
	\begin{itemize}[topsep=1mm]
		\item о понятии функций высших порядков
		\item о лямбда-исчислении
		\item о принципах построения программ высокого качества на функциональных языках
	\end{itemize}

\noindent\textbf{знать:}
	\begin{itemize}[topsep=1mm]
		\item основные парадигмы функционального программирования
		\item подходы к представлению и интерпретации функциональных программ
	\end{itemize}


% \noindent\textbf{владеть:}
% 	\begin{itemize}[topsep=1mm]
% 	\end{itemize}

\section{Содержание и структура дисциплины}
% Общая трудоемкость дисциплины составляет 4 зачетных единицы, 144 часа.
Общая трудоемкость дисциплины составляет 5 зачетных единиц, 180 часов.

\begin{center}
\begin{longtable}{|c|c|p{0.25\textwidth}|p{1.4cm}|p{1.4cm}|p{1.4cm}|p{1.4cm}|p{1.4cm}|}\hline
\multicolumn{1}{|c|}{\multirow{2}{*}{\parbox[c]{1.7cm}{\bfseriesсеместр}}} &
\multicolumn{1}{c|}{\multirow{2}{*}{\parbox[c]{1.6cm}{\bfseries~\\№\\раздела}}} &
\multicolumn{1}{c|}{\multirow{2}{*}{\parbox[c]{0.2\textwidth}{\bfseries~\\Наименование\\раздела\\дисциплины}}} &
\multicolumn{5}{c|}{\parbox[c]{0.4\textwidth}{\bfseries{}Виды учебной нагрузки и их\\ трудоемкость, часы}}\\\cline{4-8}
& & &
\multicolumn{1}{c|}{\bfseries\begin{sideways}Лекции\end{sideways}} &
\multicolumn{1}{c|}{\bfseries\begin{sideways}\parbox[c]{1.4cm}{Лабораторные~~\\работы}\end{sideways}} &
\multicolumn{1}{c|}{\bfseries\begin{sideways}СРС\end{sideways}} &
\multicolumn{1}{c|}{\bfseries\begin{sideways}\parbox[c]{0.25\textwidth}{Экзамен}\end{sideways}} &
\multicolumn{1}{c|}{\bfseries\begin{sideways}Всего часов\end{sideways}}\\\hline
7 & 1 & Понятие о функциональном программировании & 20 & 20 & 38 & & 78\\\hline
7 & 2 & Интерпретация и компиляция функциональных программ & 16 & 18 & 32 & & 66\\\hline
7 & 1, 2 & Промежуточная аттестация (экзамен) &  &  &  & 36 & \\\hline

\end{longtable}
\end{center}


% \begin{description}
% \item[Раздел 1.] «Понятие о функциональном программировании».\item[Раздел 2.] «Интерпретация и компиляция функциональных программ».
% \end{description}

\ssect{Лекции}

\begin{center}
\begin{longtable}{|c|c|c|p{0.3\textwidth}|c|}\hline
\multicolumn{1}{|c|}{\parbox{1cm}{\bfseries~\\№\\п/п\\~}} &
\multicolumn{1}{c|}{\parbox{2cm}{\bfseries №\\раздела}} &
\multicolumn{1}{c|}{\parbox{3cm}{\bfseries Объем,\\часов}} &
\multicolumn{1}{c|}{\bfseries Тема лекции} &
\multicolumn{1}{c|}{\parbox{3.5cm}{\bfseries Формируемые компетенции}} \\\hline
1 & 1 & 5 & Функциональное программирование & \compone{} \\\hline
2 & 1 & 5 & Введение в язык Haskell & \compone{} \\\hline
3 & 1 & 5 & Функции высших порядков & \compthree{} \\\hline
4 & 1 & 5 & Лямбда-исчисление & \compthree{} \\\hline
5 & 2 & 8 & Представление функциональных программ & \comptwo{} \\\hline
6 & 2 & 8 & Интерпретация & \comptwo{} \\\hline

\multicolumn{3}{|c|}{Итого:} & 36 & \\\hline
\end{longtable}
\end{center}


\ssect{Лабораторные работы}

\begin{center}
\begin{longtable}{|c|c|c|p{0.3\textwidth}|p{0.2\textwidth}|c|}\hline
\multicolumn{1}{|c|}{\parbox[c]{0.6cm}{\bfseries~\\№\\п/п\\~}} &
\multicolumn{1}{c|}{\parbox[c]{1.6cm}{\bfseries №\\раздела}} &
\multicolumn{1}{c|}{\parbox[c]{1.8cm}{\bfseries Трудо-\\ёмкость,\\часов}} &
\multicolumn{1}{c|}{\parbox[c]{3.5cm}{\bfseries Наименование лабораторной\\работы}} &
\multicolumn{1}{c|}{\parbox[c]{3cm}{\bfseries Наименование\\лаборатории}} &
\multicolumn{1}{c|}{\parbox{3.1cm}{\bfseries Формируемые компетенции}} \\\hline
1 & 1 & 5 & Функциональное программирование: проба пера & Компьютерный класс & \compone{} \\\hline
2 & 1 & 5 & Рекурсия & Компьютерный класс & \compone{} \\\hline
3 & 1 & 5 & Работа со строками & Компьютерный класс & \compthree{} \\\hline
4 & 1 & 5 & Дерево & Компьютерный класс & \compthree{} \\\hline
5 & 2 & 9 & Граф & Компьютерный класс & \comptwo{} \\\hline
6 & 2 & 9 & Лямбда-исчисление & Компьютерный класс & \comptwo{} \\\hline

\multicolumn{3}{|c|}{Итого:} & 38 & & \\\hline
\end{longtable}
\end{center}


\ssect{Самостоятельная работа студента}\\

\begin{center}
\begin{longtable}{|c|c|c|p{0.6\textwidth}|}\hline
\multicolumn{1}{|c|}{\parbox[c]{.6cm}{\bfseries~\\№\\п/п\\~}} &
\multicolumn{1}{c|}{\parbox[c]{1.7cm}{\bfseries №\\раздела}} &
\multicolumn{1}{c|}{\parbox[c]{3.1cm}{\bfseries Трудоёмкость,\\часов}} &
\multicolumn{1}{c|}{\parbox[c]{4cm}{\bfseries Вид СРС}} \\\hline
1 & 1 & 19 & Подготовка к лекциям\\\hline
2 & 1 & 19 & Выполнение лабораторных работ\\\hline
3 & 2 & 16 & Подготовка к лекциям\\\hline
4 & 2 & 16 & Выполнение лабораторных работ\\\hline

\multicolumn{2}{|c|}{Итого:} & 70 & \\\hline
\end{longtable}
\end{center}


\ssect{Домашние задания, типовые расчеты и т.п.}\\
Не предусмотрены.

\ssect{Рефераты}\\
Не предусмотрены.

\ssect{Курсовые работы по дисциплине}\\
Не предусмотрены.

\section{Образовательные технологии}

Учебный курс состоит из двух учебных модулей. По окончании каждого модуля проводятся контрольные работы в виде электронных тестов для проверки усвоения теоретического материала и в виде задач для решения на компьютере по аналогии с задачами, выданными в рамках лабораторных работ. Лабораторные работы описаны в пособии [1] п.~\ref{author-res}.

При проведении лекций и практических занятий используются следующие образовательные технологии:
\begin{itemize}
	\item мультимедийные лекции;
	\item электронные формы контроля;
	\item самотестирование студентов.
\end{itemize}

Учебный процесс базируется на концепции компетентностного обучения, ориентированного на формирование конкретного перечня профессиональных компетенций, актуализацию получаемых теоретических знаний. Развертывание компетентностной модели обучения предполагает широкое применение инновационных способов организации учебного процесса, в том числе технологий управляемого самостоятельного обучения в том числе балльно-рейтинговой системы, а также внедрение системы онлайн-поддержки внеаудиторной работы студентов.

\section{Оценочные средства для текущего контроля успеваемости и промежуточной аттестации}

\emph{Полный комплект контрольно-оценочных материалов (Фонд оценочных средств) оформлен в виде приложения к рабочей программе дисциплины.}

\section{Учебно-методическое обеспечение дисциплины}

	\ssect[Основная литература]\label{main-lit}

\begin{enumerate}
	\item Tanenbaum A., Ostin T. Structured Computer Organization / Prentice Hall; 6th edition (August 4, 2012). 800 p.\\
	Перевод: Таненбаум Э., Остин Т. Архитектура компьютера / 6-е изд.(+CD) — СПб.: Питер, 2013. — 816 с.
\end{enumerate}

	\ssect[Дополнительная литература]
\begin{enumerate}
	\item Stallings W. Computer Organization and Architecture /  Prentice Hall; 9th edition (March 11, 2012). 792 p.\\
	Перевод: Столлингс У. Структурная организация и архитектура компьютерных систем / М: Вильямс, 2002. 896 с.
\end{enumerate}

	\ssect[Список авторских методических разработок]
	\label{author-res}
\begin{enumerate}
	\item А.\,М.~Пеленицын, Н.\,Н.~Ячменёва. Методические указания к практикуму по курсу «Архитектура компьютера» [Электронный ресурс]\\
	\url{http://open-edu.sfedu.ru/node/2622}
\end{enumerate}

	\ssect[Интернет-ресурсы]\label{online-res}
\begin{enumerate}
	\item Сопроводительные материалы к учебнику [1] п.~\ref{main-lit}:
	\begin{otherlanguage}{english}
	Structured Computer Organization, 6/E
	\end{otherlanguage}
	[Электронный ресурс]\\
	\url{http://www.pearsonhighered.com/educator/product/Structured-Computer-Organization-6E/9780132916523.page}

	\item Википедия: Свободная энциклопедия, английский раздел [Электронный ресурс]\\
	\url{http://en.wikipedia.org/wiki/Main_Page}
\end{enumerate}

\section{Материально-техническое обеспечение дисциплины}
\setenumerate[1]{label={\arabic{enumi}.}}
\setenumerate[2]{label={\alph{enumii}.}}

\begin{enumerate}
\item Лекционные занятия:
\begin{enumerate}
\item аудитория, оснащенная маркерной доской, \item комплект электронных презентаций/слайдов, \item презентационная техника (проектор, экран, компьютер/ноутбук).
\end{enumerate}
\item Лабораторные работы: компьютерный класс
\begin{enumerate}
\item рабочее место преподавателя, оснащенное компьютером с доступом в Интернет, 
\item рабочие места студентов, предназначенные для работы в электронной образовательной среде. Рабочие станции студентов должны быть оснащены операционной системой GNU/Linux, текстовым редактором и компилятором/интерпретатором языка Haskell (например, GHC).
\end{enumerate}
\end{enumerate}


% \ssect[Учебно-лабораторное оборудование]
% Лекции проводятся в мультимедийном классе с презентационным оборудованием (проектором и экраном либо интерактивной доской). Лабораторные занятия проводятся в дисплейных классах с персональными компьютерами по числу, не уступающему числу студентов.

% 	\ssect[Программные средства]

% Компьютеры в дисплейных классах должны быть снабжены операционной системой GNU/Linux, желательно в виде дружественного к пользователю дистрибутива (например, Ubuntu Linux).

% На компьютерах в дисплейном классе должны быть распакованы в каталог \texttt{/bin} или \texttt{\textasciitilde/bin} три утилиты ассемблирования (as88/t88/s88) из сопроводительных материалов к учебнику Таненбаума (см. [1] в п.~\ref{online-res}). Для выполнения лабораторной работы по микропрограммированию необходимо наличие каталога Mic1MMV (с содержимым) оттуда же. Для его работы требуется виртуальная машина Java версии не ниже 1.4.

% Для редактирования кода рекомендуется иметь установленным специализированный редактор с подсветкой синтаксиса языка ассемблера, например Geany или jEdit (оба находятся в частности в репозиториях Ubuntu Linux).

% В отдельных случаях возможна работа на компьютерах под управлением операционной системы семейства Windows, однако здесь требуется дополнительная настройка утилит ассемблирования, описанная в методических указаниях~[1] п.~\ref{author-res}.


% 	\ssect[Технические и электронные средства]

% Учёт активности студентов на курсе и основные материалы размещены в системе Moodle, развёрнутой в сети университета по адресу \url{http://edu.mmcs.sfedu.ru}. В дисплейных классах требуется доступ к этому ресурсу посредством браузера актуальной версии. Для проведения контрольных работ необходимо одновременно с доступом к указанному ресурсу ограничить доступ к другим интернет- и интранет-ресурсам, что на данный момент реализовано в лаборатории ММПТ и лаборатории кафедр алгебры и дискретной математики и информатики и вычислительного эксперимента факультета.
	

\clearpage
\section{Учебная карта дисциплины}

% \input ukd_body.tex
% \input ifmo/body2.tex

\end{document}
