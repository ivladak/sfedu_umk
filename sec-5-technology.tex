{\parindent0pt
Преподавание дисциплины ведется с применением следующих видов образовательных технологий:\\
\begin{itemize}
	\item \textbf{Информационные технологии}: использование электронных образовательных ресурсов при подготовке к лекциям и лабораторным работам разделов 1 и 2.
	\item \textbf{Мультимедийные лекции}
	\item \textbf{Электронные формы контроля}: система Ejudge
\end{itemize}

%TODO: сделать работу в команде и прочую муть
% 3. Case-study, 4. Игра, 5. Проблемное обучение, 6. Контекстное обучение, 7. Обучение на основе опыта, 8. Индивидуальное обучение, 9. Междисциплинарное обучение, 10. Опережающая самостоятельная работа
}

% Учебный курс состоит из двух учебных модулей. По окончании каждого модуля проводятся контрольные работы в виде электронных тестов для проверки усвоения теоретического материала и в виде задач для решения на компьютере по аналогии с задачами, выданными в рамках лабораторных работ. Лабораторные работы описаны в пособии [1] п.~\ref{author-res}.

% При проведении лекций и практических занятий используются следующие образовательные технологии:
% \begin{itemize}
% 	\item мультимедийные лекции;
% 	\item электронные формы контроля;
% 	\item самотестирование студентов.
% \end{itemize}

% Учебный процесс базируется на концепции компетентностного обучения, ориентированного на формирование конкретного перечня профессиональных компетенций, актуализацию получаемых теоретических знаний. Развертывание компетентностной модели обучения предполагает широкое применение инновационных способов организации учебного процесса, в том числе технологий управляемого самостоятельного обучения в том числе балльно-рейтинговой системы, а также внедрение системы онлайн-поддержки внеаудиторной работы студентов.
